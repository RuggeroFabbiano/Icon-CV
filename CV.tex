% !TEX encoding = UTF-8 Unicode

% Ruggero Fabbiano
% Example template for Icon-CV package
% 30/10/2022

%%%%%%%%%%%%%%%%%%%%%%%%%%%%%%%%%%%%%%%%%%%%%%%%%%%%%%%%%%%%%%%%%%%%%%%%%%%%%%%%

\documentclass[a4paper]{article}
\usepackage{icon-cv} % options: blackwhite, greyscale

% FONT

\usepackage{tgadventor} % Gyre Adventor, a font from the LaTeX font catalogue
\renewcommand*\familydefault{\sfdefault}
\usepackage[T1]{fontenc}

% Use this to set font size outside 10-12 pt.
% \usepackage{scrextend}
% \changefontsizes{9}


%%%%%%%%%%%%%%%%%%%%%%%%%%%%%%%%%%%%%%%%%%%%%%%%%%%%%%%%%%%%%%%%%%%%%%%%%%%%%%%%

% CUSTOM (RE)DEFINITIONS

% Set your image folder
% \renewcommand\imageFolder{IconCV}

% Adapt column width (the argument value is the width percentage taken by leftmost column)
% \columnratio{0.75}

% Based on your font size, you may have to tweak the spacing between Education title and its rule
% \setlength{\EducationRuleSep}{3pt}

% Colours (DVI PS names directly available)
% \colorlet{headerColour}{pink}
% \colorlet{nameColour}{Blue}
% \colorlet{linkColour}{Aquamarine}
% \colorlet{titleColour}{black}
% \colorlet{ruleColour}{black}
% \colorlet{timeLineColour}{gray}
% \colorlet{jobColour}{Blue}
% \colorlet{skillColour}{pink}
% \colorlet{languageColour}{pink}

% Titles
% \renewcommand{\Experiences}{Work Experience}
% \renewcommand{\Education}{Education}
% \renewcommand{\Skills}{Skills}
% \renewcommand{\Languages}{Languages}
% \renewcommand{\Courses}{Courses and Certifications}

% Bullets (MathABX/FontAwesome symbols available)
% \renewcommand{\jobSep}{{\small$\bigvarstar$}}
% \renewcommand{\labelitemi}{\small$\bigvarstar$}
% \renewcommand{\diplomaSep}{{\small$\bigvarstar$}}
% \renewcommand{\courseSep}{{\small$\bigvarstar$}}

% The following commands define the symbols used to show reached levels (filled vs. empty star)
% \renewcommand{\OK}{\raisebox{-1.0pt}{\LARGE$\filledstar$}} % symbol for reached levels
% \renewcommand{\NOK}{\raisebox{-1.0pt}{\LARGE$\smallstar$}} % symbol for unreached levels

% Add any other new command here
% \newcommand{\myCommand}[1]{#1}


%%%%%%%%%%%%%%%%%%%%%%%%%%%%%%%%%%%%%%%%%%%%%%%%%%%%%%%%%%%%%%%%%%%%%%%%%%%%%%%%

% INPUT

% Personal information

\name{First name}{Last name} % required
\titles{all your titles go here} % required
\qualification{your last role / desired position etc.} % optional

\address{address}{Country} % optional
% if a country-flag file named as the Country argument exists in image folder, print it instead of
% country name

\nationality[UK]{nationality} % optional
% the optional argument is the name of the nationality country-flag file that, if provided, will
% replace the nationality icon

\license{driving licenses} % optional
\phone{telephone number} % optional
\eMail{e-mail address} % optional
\LinkedIn{LinkedIn alias} % optional (creates full link from profile name)
\GitHub{GitHub alias} % optional (creates full link from profile name)


% Work experience
% list your experiences in chronological order here, they will be printed starting from the last one

\experience{Company}{location}{Country}{role}{Subject}{period}
% if a file named as the Company argument exists in image folder, prints it instead of company name
% if a country-flag file named as the Country argument exists in image folder, prints it instead of
% country name

\resume{%\\
    This can be an exhaustive work-experience description; you can also use bulleted lists and
    other environments here inside.    
    For an optimal paragraph spacing the initial new line is not set, so if you start with a simple
    paragraph and are not using an environment (\textit{e.g.}, \texttt{itemize}), you may have to
    add a new line first, as the comment you see at the beginning of this \texttt{resume}.
}

% You can repeat the experience-resume commands as much as you need


% Education

\diploma{Degree}{Major}{University}{Country}{year}
% if a country-flag file named as the Country argument exists in image folder, prints it instead of
% country name


% Skills

\skill[LaTeX]{Group 1}{Skill 1}{1} % the optional argument prints a related logo (must be present)
% "Group" is a sub-title for a set of related skills; it is only printed once at the top
% it can be omitted after the first declaration
% the last argument is a number from 1 to 5, indicating the level of mastering of such skill 
\skill{Group 2}{Skill 2}{2} 
\skill[Git]{}{Skill 3}{3}
\skill{Group 3}{Skill 4}{4}


% Languages

\newLanguage[10]{Language}{5} % optional argument indicates total length of level bar (default: 100)
\newLanguage{UK}{50}
% if a flag filge called as Language argument exists, it is printed instead of Language name
% last argument is the level value


% Courses and certifications

\course[LaTeX]{Title}{\LaTeX school} % optional argument provides a logo for the course provider
% the last argument is the full name of the course provider; it can also be an extended logo with
% the full name of the institution (that must be in the image folder), or a plain name
\course[LaTex]{Another LaTeX class}{LaTeX_project}


%%%%%%%%%%%%%%%%%%%%%%%%%%%%%%%%%%%%%%%%%%%%%%%%%%%%%%%%%%%%%%%%%%%%%%%%%%%%%%%%

\makeCV % print the CV!